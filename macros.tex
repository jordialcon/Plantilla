%=========================TAULES============================================================================================
% Es crea una nova comanda per insertar una taula. En aquest cas estara dintre de la carpeta taules. Per tal de fer servir la comanda \taula{Nom del fitxer}{amplada} on amplada pot ser \linewidth que es l'amplada de la linia, es pot posar 0.7\linewidth pq sigui un 70% de l'amplada.
\newcommand{\taula}[2]{\LTXtable{#2}{tables/#1}} 
%\renewcommand{\tablename}{Tabla} %per posar tabla al caption de les taules

%=========================INSERTAR IMATGES==================================================================================
%Aquesta macro inserta una imatge centrada. Per tal de colocar la imatge s'ha de fer servir la comanda:
%	\imatge{Nom fitxer dintre carpeta imatges}{Titol de la imatge}{Amplada}
%	el \label queda definit com el nom del fitxer
\newcommand{\imatge}[3]{
 \begin{figure}[htb]
  \begin{center}
   %\includegraphics[width=#3cm,angle=-90]{#1}
   \includegraphics[width=#3\linewidth]{#1}
   \caption{#2}
   \label{fig:#1}
  \end{center}
 \end{figure}
}

\newcommand{\imatgeR}[4]{
 \begin{figure}[htb]
  \begin{center}
   \includegraphics[width=#3\linewidth,angle=#4]{#1}
   \caption{#2}
   \label{fig:#1}
  \end{center}
 \end{figure}
}

\hypersetup{
    colorlinks,
    linkcolor={blue!80!black},
    citecolor={blue!80!black},
    urlcolor={blue!80!black}
}

% Espaiat a l'index
%\setlength{\cftbeforechapskip}{0.1ex} % Nomes al report i altres amb \chapter
\setlength{\cftbeforesecskip}{0.5ex}

% Portada

\newcommand{\authorname}{\textbf}
\newcommand{\authorgroup}{\emph}
\newcommand{\authoraddress}{\emph}
\newcommand{\authormail}{\emph}

\makeatletter
\def\@maketitle{%
  \newpage
  \null
%  \vskip 2em%
  \begin{center}%
  \let \footnote \thanks
    {\LARGE\MakeUppercase \@title \par}%
    \vskip 1.0em%
    {\Large\subject{}}
    \vskip 1.0em%
    {\normalsize
      \lineskip .5em%
      \begin{tabular}[t]{c}%
        \@author
      \end{tabular}\par}%
    \vskip -0.5em%  
    {\large\bf Group: \group{}}
    \vskip 1em%
%    \vfill
    {\large \@date, \city{}}%
  \end{center}%
  \par
  \vskip -0.5em}
\makeatother

%Maketitle Conf______________________________________________________
%\pretitle{\begin{center}\Large\MakeUppercase}
%\posttitle{\\\subject{}\end{center}}

%\preauthor{\begin{center}\normalsize \lineskip 0.5em\begin{tabular}[t]{c}}
%\postauthor{\end{tabular}\par\end{center}}

%\predate{\begin{center}\normalsize}
%\postdate{, \city{}\par\end{center}}
%_____________________________________________________________________


% Headers___________________________________________________________

\pagestyle{fancy}
\fancyhead[L]{\subject{}, ESEIAAT, Group \group{}}
\fancyhead[R]{\thedate{}, \city{}}

\setlength{\headheight}{15pt} 

%___________________________________________________________________

\DeclarePairedDelimiter{\norm}{\lVert}{\rVert}

\newcommand{\vnorm}[1]{\norm{\vec{#1}}}

\sloppy % better line breaks

\setlength{\nomitemsep}{-\parsep} % Nomenclature spacing

\newcommand{\fref}[1]{Fig.~\ref{#1}}
\newcommand{\tref}[1]{Tab.~(\ref{#1})}
\newcommand{\eref}[1]{Eq.~(\ref{#1})}
%\newcommand{\cref}[1]{Cap.~\ref{#1} \nameref{#1}} % \chapter references not available in article
\newcommand{\sref}[1]{Sec.~\ref{#1} \nameref{#1}}
\newcommand{\aref}[1]{Ap.~(\ref{#1})}
\newcommand{\alref}[1]{Algorithm~\ref{#1}}
\newcommand{\ejref}[1]{Ex.~(\ref{#1})}

\DeclareSIUnit\curie{Ci}