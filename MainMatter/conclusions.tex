\section{Conclusions}\label{sec:conclusions}

Throughout the assignment it can be noted that it is more efficient to exit the Earth's gravitational influence by means of an hyperbolic trajectory rather than a parabollic one. That is because in the latter, a second speed increment is needed, hence making the total speed increment larger.

Then, to reach Jupiter, it can be shown that, eventhough it takes much longer to reach Jupiter with a fly-by maneuver, both the initial speed increment and the needed one to describe an orbit around Jupiter with the same orbital radius than Europa, are significantly lower. When it comes to space missions, saving fuel is prioritised, so it can be concluded that the best way to reach Jupiter will be by means of the fly-by maneuver.  

During the document, it is assumed that $r_{SOI} = 0$ for all celestial bodies, so the vechicle is only under the influence of just one gravitational field at a point. Besides, it has not been taken into consideration the event of possible collisions between celestial bodies.

All in all, this is a preliminary study that cannot be used to describe the exact trajectory that the vehicle has to follow -it would be affected by the gravitational fields of other celestial bodies apart from the Sun, the Earth and Jupiter- but it presents a good first approximation and estimation of the initial velocity increment that is needed to reach such destination. 
