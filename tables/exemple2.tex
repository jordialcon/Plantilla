% Per separar les columnes es fa servir el & i per canviar a la fila inferior el \\
% Si voleu colocar una linea horitzontal heu de colocar \hline despres de la fila on volgueu la linea

\renewcommand{\tablename}{Taula}
                  %%%%%%%  Tantes X (majuscules) com columnes. Els | son per les linies verticals (AltGrf+1)
\begin{longtable}{|X|X|X|}
         %%%%%%%%%%%
\caption{Titolet}   %Introduir el titol de la taula
       %%%%%%%%%
\label{fran}\\      %Etiqueta, serveix per fer referencies creuades, posar el nom del fitxer

       %%%%%%%%%%%%%%% Copy paste de dalt a baix, es la capalera de columna. Si en cal una de xunga aviseu.
\hline the & head& line\\ \hline \endfirsthead
\hline test & test & test \\
\hline the & head& line\\ \hline \endhead

                    % El 3 son el numero de columnes
\hline \multicolumn{3}{|r|}{\scriptsize{Continua a la pgina segent $\hookrightarrow$}}\\ \hline \endfoot
\hline \endlastfoot

a & b & c\\
a & b & c\\
a & b & c\\
a & b & c\\
a & b & c\\
a & b & c\\
a & b & c\\
a & b & c\\
a & b & c\\
a & b & c\\
a & b & c\\
a & b & c\\
a & b & c\\
a & b & c\\
a & b & c\\
a & b & c\\
a & b & c\\
a & b & c\\
a & b & c\\
a & b & c\\
a & b & c\\
a & b & c\\
a & b & c\\
a & b & c\\
a & b & c\\
a & b & c\\
a & b & c\\
a & b & c\\
a & b & c\\
a & b & c\\
a & b & c\\
a & b & c\\
a & b & c\\
a & b & c\\
a & b & c\\
a & b & c\\
a & b & c\\
a & b & c\\
a & b & c\\
a & b & c\\
a & b & c\\
a & b & c\\
a & b & c\\
a & b & c\\
a & b & c\\
a & b & c\\
a & b & c\\
\end{longtable}
